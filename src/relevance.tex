\section{Relevance}

\subsection{Spread}

\begin{frame}{Popularity}
	\begin{itemize}
	\item Running on:
		\begin{itemize}
		\item Major distros: Fedora, Debian, Ubuntu, ...
		\item Android
		\item Virtually any system with Linux >= 2.6.26
		\end{itemize}
	\item Integrated with high(er) level tools:
		\begin{itemize}
		\item \url{docker.io} - The Linux Container Runtime
		\item \url{libvirt.org} - The Virtualization API
		\item \url{criu.org} - Checkpoint-Restart in Userspace
		\end{itemize}
	\item Maintained by both kernel and userspace developers
	\end{itemize}
\end{frame}

\subsection{Use Cases}

\begin{frame}{Use Cases}
	\begin{itemize}
	\item General:
		\begin{itemize}
		\item Server replication
		\item Application sandboxing
		\item Legacy software support
		\item Live migration
		\item GPL insulation
		\end{itemize}
	\item Embedded (networking, smartphones):
		\begin{itemize}
		\item Separate traffic from different departments
		\item Separate QoS policies
		\item Run RTOS and HLOS at the same time
		\end{itemize}
	\end{itemize}
\end{frame}

\begin{frame}{Freescale USDPAA in Containers}
	\begin{itemize}
	\item DPAA - DataPath Acceleration Architecture
		\begin{itemize}
		\item HW architecture providing advanced networking capabilities
		\item Present in dedicated networking equipment
		\item Traffic shaping, package accelerators, cryptography engine
		\end{itemize}
	\item USDPAA - User Space DPAA
		\begin{itemize}
		\item Userspace drivers based on the uio framework
		\item Increased flexibility in application development
		\item Reduced risk in bugging in kernel
		\item Better error handling and system protection
		\item Performance overhead
		\end{itemize}
	\item Multiple USDPAA instances in containers
		\begin{itemize}
		\item Improved isolation
		\item Additional protection layer
		\item Finer resource tuning
		\end{itemize}
	\end{itemize}
\end{frame}